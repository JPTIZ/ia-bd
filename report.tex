%-------------------------------------------------------------------------------
\documentclass[answers]{exam}

%-------------------------------------------------------------------------------
% Packages
\usepackage{amsmath}
\usepackage[portuguese]{babel}

%-------------------------------------------------------------------------------
% User-commands
\newcommand{\todo}[1]{{\color{red}{#1}}}

%-------------------------------------------------------------------------------
% Project configs
\title{Relatório de I.A.: OWL, Parte 1}
\author{Cauê Baasch de Souza \\
        João Paulo Taylor Ienczak Zanette}
\date{\today}

%-------------------------------------------------------------------------------
\begin{document}
    \maketitle{}

    \begin{center}
        \fbox{\setlength{\fboxsep}{1em}\fbox{\parbox{5.5in}{%
            \textbf{Observações:}

            \begin{itemize}
                \item O resultado da parte 1 deve ser um pequeno texto
                    explicando o \textbf{entendimento} de vocês sobre os
                    tópicos sugeridos. Portanto, o texto deve ter \textbf{no
                    máximo} 2 páginas;
                \item Citar todas as fontes utilizadas para a pesquisa
                    (Wikipédia também serve).
            \end{itemize}
        }}}
    \end{center}


    \begin{questions}
        \question{}
        Em OWL 2, qual é a diferença entre os axiomas de class
        \texttt{subClassOf} e \texttt{equivalentTo}? Apresente as definições de
        cada um e exemplos de uso dos dois, dentro do domínio escolhido pela
        dupla para a parte prática. Descreva especialmente a diferença dos
        axiomas de classe quanto às inferências possíveis, ou seja, teste os
        exemplos \underline{no seu domínio} e descreva as inferências.

        \begin{solution}
        \end{solution}

        \question{}
        Compare a lógica descritiva que fundamenta a OWL 2 (na sua variação
        mais expressiva) com lógica de 1ª ordem. Apresente um exemplo do que é
        possível expressar com lógica de 1ª ordem que não conseguimos com
        lógica descritiva.

        \begin{solution}
        \end{solution}
    \end{questions}
\end{document}
