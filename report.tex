%-------------------------------------------------------------------------------
\documentclass[answers]{exam}

%-------------------------------------------------------------------------------
% Packages
\usepackage{amsmath}
\usepackage[portuguese]{babel}
\usepackage{hyperref}
\usepackage{xcolor}

%-------------------------------------------------------------------------------
% User-commands
\newcommand{\todo}[1]{{\color{red}{#1}}}

%-------------------------------------------------------------------------------
% Project configs
\title{Relatório de I.A.: OWL, Parte 1}
\author{Cauê Baasch de Souza \\
        João Paulo Taylor Ienczak Zanette}
\date{\today}

%-------------------------------------------------------------------------------
\begin{document}
    \maketitle{}

    \todo{%
        TO-DO\@:
        \begin{itemize}
            \item Adicionar enunciado da 2ª parte (parte prática);
            \item Escolher domínio dos exemplos.
        \end{itemize}
    }

    \begin{center}
        \fbox{\setlength{\fboxsep}{1em}\fbox{\parbox{5.5in}{%
            \textbf{Observações:}

            \begin{itemize}
                \item O resultado da parte 1 deve ser um pequeno texto
                    explicando o \textbf{entendimento} de vocês sobre os
                    tópicos sugeridos. Portanto, o texto deve ter \textbf{no
                    máximo} 2 páginas;
                \item Citar todas as fontes utilizadas para a pesquisa
                    (Wikipédia também serve).
            \end{itemize}
        }}}
    \end{center}


    \begin{questions}
        \question{}
        Em OWL 2, qual é a diferença entre os axiomas de class
        \texttt{subClassOf} e \texttt{equivalentTo}? Apresente as definições de
        cada um e exemplos de uso dos dois, dentro do domínio escolhido pela
        dupla para a parte prática. Descreva especialmente a diferença dos
        axiomas de classe quanto às inferências possíveis, ou seja, teste os
        exemplos \underline{no seu domínio} e descreva as inferências.

        \begin{solution}
            \begin{description}
                \item [\texttt{subClassOf}:] Dadas duas classes A e B e
                    individuos que pertençam a apenas uma delas. Ao se fazer
                    \texttt{subClassOf(:A :B)}, se está indicando que todo
                    elemento pertencente a A também pertence a B, \textbf{mas o
                    contrário não necessariamente}. Sendo assim, se os
                    elementos ${1, 2, 3}$ pertencerem a A e ${4, 5, 6}$ a B,
                    \texttt{subClassOf(:A :B)} fará com que os indivíduos que
                    pertencem a A e B sejam, respectivamente, ${1, 2, 3}$ e
                    ${1, 2, 3, 4, 5, 6}$;
                \item [\texttt{equivalentTo}:] Dadas duas classes A e B e
                    indivíduos que pertençam a apenas uma delas. Ao se fazer
                    \texttt{equivalentTo}, se está indicando que todo elemento
                    pertencente a A também pertence a B, \textbf{e vice-versa}.
                    Sendo assim, se os elementos ${1, 2, 3}$ pertencerem a A e
                    ${4, 5, 6}$ a B, \texttt{equivalentTo(:A :B)} fará com que
                    tanto A quanto B sejam compostas pelos indivíduos ${1, 2,
                    3, 4, 5, 6}$.
            \end{description}

            \todo{%
                Exemplificar com elementos do domínio escolhido na parte
                prática.
            }
        \end{solution}

        \question{}
        Compare a lógica descritiva que fundamenta a OWL 2 (na sua variação
        mais expressiva) com lógica de 1ª ordem. Apresente um exemplo do que é
        possível expressar com lógica de 1ª ordem que não conseguimos com
        lógica descritiva.

        \begin{solution}
        \end{solution}
    \end{questions}

    \bibliographystyle{unsrt}
    \bibliography{refs}
    \nocite{*}
\end{document}


% A class axiom may contain (multiple) owl:equivalentClass statements.
% owl:equivalentClass is a built-in property that links a class description to
% another class description. The meaning of such a class axiom is that the two
% class descriptions involved have the same class extension (i.e., both class
% extensions contain exactly the same set of individuals).
