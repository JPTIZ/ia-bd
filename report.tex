%-------------------------------------------------------------------------------
\documentclass[answers]{exam}

%-------------------------------------------------------------------------------
% Packages
\usepackage{amsmath}
\usepackage[portuguese]{babel}
\usepackage{environ}
\usepackage{hyperref}
\usepackage{minted}
\usepackage{xcolor}

%-------------------------------------------------------------------------------
% User-commands
\newcommand{\todo}[1]{{\color{red}{#1}}}

\NewEnviron{superframe}{%
    \begin{center}
        \fbox{\setlength{\fboxsep}{1em}\fbox{\parbox{5.5in}{%
            \BODY{}
        }}}
    \end{center}
}

\NewEnviron{answer}{%
    \begin{samepage}
        \begin{solution}
            \BODY{}
        \end{solution}
    \end{samepage}
}

\newmintedfile[textfile]{text}{autogobble, breaklines}

\newcommand{\result}[2]{\textfile[firstline=#1, lastline=#2]{thesampleoutput}}
\newcommand{\ontology}[2]{\textfile[firstline=#1, lastline=#2]{lang-ontology.owl}}

% PT-BR for exam
\vqword{Questão:}
\renewcommand{\solutiontitle}{\noindent\textbf{Resposta:}\par\noindent}

\unframedsolutions{}

%-------------------------------------------------------------------------------
% Project configs
\title{Relatório de I.A.: OWL, Parte 1}
\author{Cauê Baasch de Souza \\
        João Paulo Taylor Ienczak Zanette}
\date{\today}

%-------------------------------------------------------------------------------
\begin{document}
    \maketitle{}

    \todo{%
        TO-DO\@:
        \begin{itemize}
            \item Adicionar enunciado da 2ª parte (parte prática);
            \item Escolher domínio dos exemplos.
        \end{itemize}
    }

    \section{Geral}

    \begin{superframe}
        \begin{itemize}
            \item Será necessário entregar com a parte 1 e uma breve descrição
                do domínio escolhido, bem como das principais classes e
                propriedades definidas, ontologias importadas (quando
                aplicável), e testes realizados com os indivíduos (o que foi
                inferido a partir da terminologia);

            \item Além disso, o arquivo RDF/OWL com o modelo conceitual.
        \end{itemize}
    \end{superframe}

    \section{Decisões do trabalho}

    \subsection{Domínio escolhido}

    Linguagens, suas implementações e programas descritos nelas.

    \subsection{Principais Classes:}

    \begin{itemize}
        \item \textbf{Language:} Representa linguagens de programação (e.g.
            Python, Ruby, C\ldots);
        \item \textbf{Implementation:} Representa uma implementação de uma
            linguagem de programação. Por exemplo: CPython, Pypy e MicroPython
            são implementações de Python;
        \item \textbf{Compiler:} Representa um compilador, podendo incluir
            JITs. Identifica-se um compilador como AOT (\textit{Ahead of Time})
            pela propriedade \texttt{aot}. A negativa dessa propriedade
            identifica o compilador como JIT\@.
        \item \textbf{Interpreter:} Representa um interpretador de uma
            linguagem.
    \end{itemize}

    \section{Parte 1: Pesquisa Teórica}

    \begin{superframe}
        \textbf{Observações:}

        \begin{itemize}
            \item O resultado da parte 1 deve ser um pequeno texto
                explicando o \textbf{entendimento} de vocês sobre os
                tópicos sugeridos. Portanto, o texto deve ter \textbf{no
                máximo} 2 páginas;
            \item Citar todas as fontes utilizadas para a pesquisa
                (Wikipédia também serve).
        \end{itemize}
    \end{superframe}

    \begin{questions}
        \question{}
        Em OWL 2, qual é a diferença entre os axiomas de class
        \texttt{subClassOf} e \texttt{equivalentTo}?
        \begin{itemize}
            \item Apresente as definições de cada um e exemplos de uso dos
                dois, dentro do domínio escolhido pela dupla para a parte
                prática.
            \item Descreva especialmente a diferença dos axiomas de classe
                quanto às inferências possíveis, ou seja, teste os exemplos
                \underline{no seu domínio} e descreva as inferências.
        \end{itemize}

        \clearpage{}

        \begin{answer}
            \textbf{\texttt{subClassOf}:}

            Dadas duas classes A e B e individuos que pertençam a apenas uma
            delas. Ao se fazer \texttt{subClassOf(:A :B)}, se está indicando
            que todo elemento pertencente a A também pertence a B, \textbf{mas
            o contrário não necessariamente}, ou seja: $A \subseteq B$.  Sendo
            assim, para $\{1, 2, 3\} \in A$ e $\{4, 5, 6\} \in B$,
            \texttt{subClassOf(:A :B)} fará com que os indivíduos que pertencem
            a A e B sejam, respectivamente, $\{1, 2, 3\}$ e $\{1, 2, 3, 4, 5,
            6\}$.

            \textbf{Exemplo:}

            No domínio escolhido para a parte prática, foi definido que toda
            \textbf{Máquina Virtual} (VM) é também um \textbf{Programa}. Porém,
            faz sentido que nem todo programa seja uma VM\@: nesse caso, não é
            possível inferir, por exemplo, que o GCC seja uma VM mesmo que seja
            um programa.  Porém, é possível inferir que o GCC é um programa
            simplesmente por defini-lo como um compilador, afinal todo
            compilador é também um programa.

            Na prática, tem-se as seguintes definições de classes em OWL2:

            \ontology{23}{25}
            \ontology{35}{37}

            E a instância ``JVM'' foi definida da forma:

            \ontology{105}{107}

            Partindo dessa instância (e de outras VMs), é possível ver que, ao
            carregá-las para o leitor de OWL2, elas são atribuídas à classe
            \texttt{Program}:

            \result{15}{17}

            É perceptível que, apesar de não declaradas explicitamente como
            programas, as instâncias foram inferidas como sendo da classe
            \texttt{Program}.

            \textbf{\texttt{equivalentTo}:}

            Dadas duas classes A e B e indivíduos que pertençam a apenas uma
            delas. Ao se fazer \texttt{equivalentTo}, se está indicando que
            todo elemento pertencente a A também pertence a B, \textbf{e
            vice-versa}.  Sendo assim, para $\{1, 2, 3\} \in A$ e $\{4, 5, 6\}
            \in B$, \texttt{equivalentTo(:A :B)} fará com que tanto A quanto B
            sejam compostas pelos indivíduos $\{1, 2, 3, 4, 5, 6\}$.

            \textbf{Exemplo:}

            Foi definido que existem \textbf{Linguagens} e \textbf{Dialetos}. É
            fato que todo dialeto também é uma linguagem de programação, visto
            que tem um compilador próprio, não necessariamente um programa
            escrito em um dialeto compila na linguagem na qual foi baseado,
            além de tem suas próprias especificidades. Da mesma forma, a
            descrição de uma linguagem de programação descreve também um
            dialeto (como uma espécie de ``dialeto-pai''). Porém, ainda assim,
            nem um nem outro são o mesmo conceito: dialetos são variações de
            uma linguagem, enquanto que linguagens são linguagens como ampla
            definição.  Sendo assim, qualquer elemento descrito como linguagem
            será inferido como dialeto e vice-versa.
        \end{answer}

        \question{}
        Compare a Lógica Descritiva que fundamenta a OWL 2 (na sua variação
        mais expressiva) com lógica de 1ª ordem. Apresente um exemplo do que é
        possível expressar com lógica de 1ª ordem que não conseguimos com
        lógica descritiva.

        \begin{answer}
            Na prática, DLs (\textit{Descriptive Logics}) geralmente podem ser
            vistas como subconjuntos \textbf{decidíveis} de FOL
            (\textit{First-Order Logic})~\cite{Tsarkov2003DLRV}. Portanto, todo
            problema descrito com FOL é também descritível com DL, porém o
            contrário não é verdade: FOLs que descrevem problemas indecidíveis
            não possuem uma descrição em determinadas DLs.

            Da parte de OWL 2, esta é uma DL cuja expressividade é semelhante à
            de $\mathcal{SROIQ^{(D)}}$ (nomenclatura que se dá devido às
            descrições de DL fazem parte de OWL 2 --- e.g. $\mathcal{I}$ indica
            que OWL 2 inclui propriedades de inversão; e $\mathcal{O}$ indica
            que inclui nominais tais como \texttt{owl:hasValue}, que impõem
            restrições aos valores dos
            objetos~\cite{wikipedia:description-logic}).


        \end{answer}
    \end{questions}

    \section{Parte 2: Prática}
    \begin{questions}
        \question{}
        Desenvolver um modelo conceitual (ontologia de domínio) utilizando
        OWL\@.

        Construam uma representação de domínio (utilizando \emph{classes},
        \emph{relações} e \emph{indivíduos}) de alguma área de interesse
        (pesquisa em computação, temas dos TCCs, filmes, esportes, músicas,
        hobbies, etc.).

        \begin{superframe}
            \begin{itemize}
                \item A avaliação considera principalmente a utilização dos
                    axiomas de propriedades e de classes. Quanto mais
                    restrições forem modeladas, melhor;

                \item Dentre as restrições, utilize no mínimo alguma de
                    cardinalidade qualificada (\texttt{min}, \texttt{max},
                    \ldots) em alguma propriedade de objeto (ver na
                    documentação da OWL, disponível no Moodle);

                \item Outro aspecto importante é a definição dos membros das
                    classes. Utilizem suas definições para testar a definição
                    dos axiomas e também para testar o funcionamento do
                    mecanismo de inferência da linguagem.
            \end{itemize}
        \end{superframe}
    \end{questions}

    \bibliographystyle{unsrt}
    \bibliography{refs}
    \nocite{*}
\end{document}


% A class axiom may contain (multiple) owl:equivalentClass statements.
% owl:equivalentClass is a built-in property that links a class description to
% another class description. The meaning of such a class axiom is that the two
% class descriptions involved have the same class extension (i.e., both class
% extensions contain exactly the same set of individuals).
